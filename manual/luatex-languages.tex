% language=uk

\environment luatex-style

\startcomponent luatex-languages

\startchapter[reference=languages,title={Languages, characters, fonts and glyphs}]

\startsection[title={Introduction}]

\topicindex {languages}

\LUATEX's internal handling of the characters and glyphs that eventually become
typeset is quite different from the way \TEX82 handles those same objects. The
easiest way to explain the difference is to focus on unrestricted horizontal mode
(i.e.\ paragraphs) and hyphenation first. Later on, it will be easy to deal
with the differences that occur in horizontal and math modes.

In \TEX82, the characters you type are converted into \type {char} node records
when they are encountered by the main control loop. \TEX\ attaches and processes
the font information while creating those records, so that the resulting \quote
{horizontal list} contains the final forms of ligatures and implicit kerning.
This packaging is needed because we may want to get the effective width of for
instance a horizontal box.

When it becomes necessary to hyphenate words in a paragraph, \TEX\ converts (one
word at time) the \type {char} node records into a string by replacing ligatures
with their components and ignoring the kerning. Then it runs the hyphenation
algorithm on this string, and converts the hyphenated result back into a \quote
{horizontal list} that is consecutively spliced back into the paragraph stream.
Keep in mind that the paragraph may contain unboxed horizontal material, which
then already contains ligatures and kerns and the words therein are part of the
hyphenation process.

Those \type {char} node records are somewhat misnamed, as they are glyph
positions in specific fonts, and therefore not really \quote {characters} in the
linguistic sense. There is no language information inside the \type {char} node
records at all. Instead, language information is passed along using \type
{language whatsit} nodes inside the horizontal list.

In \LUATEX, the situation is quite different. The characters you type are always
converted into \nod {glyph} node records with a special subtype to identify them
as being intended as linguistic characters. \LUATEX\ stores the needed language
information in those records, but does not do any font|-|related processing at
the time of node creation. It only stores the index of the current font and a
reference to a character in that font.

When it becomes necessary to typeset a paragraph, \LUATEX\ first inserts all
hyphenation points right into the whole node list. Next, it processes all the
font information in the whole list (creating ligatures and adjusting kerning),
and finally it adjusts all the subtype identifiers so that the records are \quote
{glyph nodes} from now on.

\stopsection

\startsection[title={Characters, glyphs and discretionaries},reference=charsandglyphs]

\topicindex {characters}
\topicindex {glyphs}
\topicindex {hyphenation}

\TEX82 (including \PDFTEX) differentiates between \type {char} nodes and \type
{lig} nodes. The former are simple items that contained nothing but a \quote
{character} and a \quote {font} field, and they lived in the same memory as
tokens did. The latter also contained a list of components, and a subtype
indicating whether this ligature was the result of a word boundary, and it was
stored in the same place as other nodes like boxes and kerns and glues.

In \LUATEX, these two types are merged into one, somewhat larger structure called
a \nod {glyph} node. Besides having the old character, font, and component
fields there are a few more, like \quote {attr} that we will see in \in {section}
[glyphnodes], these nodes also contain a subtype, that codes four main types and
two additional ghost types. For ligatures, multiple bits can be set at the same
time (in case of a single|-|glyph word).

\startitemize
    \startitem
        \type {character}, for characters to be hyphenated: the lowest bit
        (bit 0) is set to 1.
    \stopitem
    \startitem
        \nod {glyph}, for specific font glyphs: the lowest bit (bit 0) is
        not set.
    \stopitem
    \startitem
        \type {ligature}, for constructed ligatures bit 1 is set.
    \stopitem
    \startitem
        \type {ghost}, for so called \quote {ghost objects} bit 2 is set.
    \stopitem
    \startitem
        \type {left}, for ligatures created from a left word boundary and for
        ghosts created from \lpr {leftghost} bit 3 gets set.
    \stopitem
    \startitem
        \type {right}, for ligatures created from a right word boundary and
        for ghosts created from \lpr {rightghost} bit 4 is set.
    \stopitem
\stopitemize

The \nod {glyph} nodes also contain language data, split into four items that
were current when the node was created: the \prm {setlanguage} (15~bits), \prm
{lefthyphenmin} (8~bits), \prm {righthyphenmin} (8~bits), and \prm {uchyph}
(1~bit).

Incidentally, \LUATEX\ allows 16383 separate languages, and words can be 256
characters long. The language is stored with each character. You can set
\prm {firstvalidlanguage} to for instance~1 and make thereby language~0
an ignored hyphenation language.

The new primitive \lpr {hyphenationmin} can be used to signal the minimal length
of a word. This value is stored with the (current) language.

Because the \prm {uchyph} value is saved in the actual nodes, its handling is
subtly different from \TEX82: changes to \prm {uchyph} become effective
immediately, not at the end of the current partial paragraph.

Typeset boxes now always have their language information embedded in the nodes
themselves, so there is no longer a possible dependency on the surrounding
language settings. In \TEX82, a mid|-|paragraph statement like \type {\unhbox0}
would process the box using the current paragraph language unless there was a
\prm {setlanguage} issued inside the box. In \LUATEX, all language variables
are already frozen.

In traditional \TEX\ the process of hyphenation is driven by \type {lccode}s. In
\LUATEX\ we made this dependency less strong. There are several strategies
possible. When you do nothing, the currently used \type {lccode}s are used, when
loading patterns, setting exceptions or hyphenating a list.

When you set \prm {savinghyphcodes} to a value greater than zero the current set
of \type {lccode}s will be saved with the language. In that case changing a \type
{lccode} afterwards has no effect. However, you can adapt the set with:

\starttyping
\hjcode`a=`a
\stoptyping

This change is global which makes sense if you keep in mind that the moment that
hyphenation happens is (normally) when the paragraph or a horizontal box is
constructed. When \prm {savinghyphcodes} was zero when the language got
initialized you start out with nothing, otherwise you already have a set.

When a \lpr {hjcode} is greater than 0 but less than 32 is indicates the
to be used length. In the following example we map a character (\type {x}) onto
another one in the patterns and tell the engine that \type {œ} counts as one
character. Because traditionally zero itself is reserved for inhibiting
hyphenation, a value of 32 counts as zero.

Here are some examples (we assume that French patterns are used):

\starttabulate[||||]
\NC                                  \NC \type{foobar} \NC \type{foo-bar} \NC \NR
\NC \type{\hjcode`x=`o}              \NC \type{fxxbar} \NC \type{fxx-bar} \NC \NR
\NC \type{\lefthyphenmin3}           \NC \type{œdipus} \NC \type{œdi-pus} \NC \NR
\NC \type{\lefthyphenmin4}           \NC \type{œdipus} \NC \type{œdipus}  \NC \NR
\NC \type{\hjcode`œ=2}               \NC \type{œdipus} \NC \type{œdi-pus} \NC \NR
\NC \type{\hjcode`i=32 \hjcode`d=32} \NC \type{œdipus} \NC \type{œdipus}  \NC \NR
\NC
\stoptabulate

Carrying all this information with each glyph would give too much overhead and
also make the process of setting up these codes more complex. A solution with
\type {hjcode} sets was considered but rejected because in practice the current
approach is sufficient and it would not be compatible anyway.

Beware: the values are always saved in the format, independent of the setting
of \prm {savinghyphcodes} at the moment the format is dumped.

A boundary node normally would mark the end of a word which interferes with for
instance discretionary injection. For this you can use the \prm {wordboundary}
as a trigger. Here are a few examples of usage:

\startbuffer
    discrete---discrete
\stopbuffer
\typebuffer \startnarrower \dontcomplain \hsize 1pt \getbuffer \par \stopnarrower
\startbuffer
    discrete\discretionary{}{}{---}discrete
\stopbuffer
\typebuffer \startnarrower \dontcomplain \hsize 1pt \getbuffer \par \stopnarrower
\startbuffer
    discrete\wordboundary\discretionary{}{}{---}discrete
\stopbuffer
\typebuffer \startnarrower \dontcomplain \hsize 1pt \getbuffer \par \stopnarrower
\startbuffer
    discrete\wordboundary\discretionary{}{}{---}\wordboundary discrete
\stopbuffer
\typebuffer \startnarrower \dontcomplain \hsize 1pt \getbuffer \par \stopnarrower
\startbuffer
    discrete\wordboundary\discretionary{---}{}{}\wordboundary discrete
\stopbuffer
\typebuffer \startnarrower \dontcomplain \hsize 1pt \getbuffer \par \stopnarrower

We only accept an explicit hyphen when there is a preceding glyph and we skip a
sequence of explicit hyphens since that normally indicates a \type {--} or \type
{---} ligature in which case we can in a worse case usage get bad node lists
later on due to messed up ligature building as these dashes are ligatures in base
fonts. This is a side effect of separating the hyphenation, ligaturing and
kerning steps.

The start and end of a sequence of characters is signalled by a \nod {glue}, \nod
{penalty}, \nod {kern} or \nod {boundary} node. But by default also a \nod
{hlist}, \nod {vlist}, \nod {rule}, \nod {dir}, \nod {whatsit}, \nod {ins}, and
\nod {adjust} node indicate a start or end. You can omit the last set from the
test by setting \lpr {hyphenationbounds} to a non|-|zero value:

\starttabulate[|c|l|]
\DB value    \BC behaviour \NC \NR
\TB
\NC \type{0} \NC not strict \NC \NR
\NC \type{1} \NC strict start \NC \NR
\NC \type{2} \NC strict end \NC \NR
\NC \type{3} \NC strict start and strict end \NC \NR
\LL
\stoptabulate

The word start is determined as follows:

\starttabulate[|l|l|]
\DB node      \BC behaviour \NC \NR
\TB
\BC boundary  \NC yes when wordboundary \NC \NR
\BC hlist     \NC when hyphenationbounds 1 or 3 \NC \NR
\BC vlist     \NC when hyphenationbounds 1 or 3 \NC \NR
\BC rule      \NC when hyphenationbounds 1 or 3 \NC \NR
\BC dir       \NC when hyphenationbounds 1 or 3 \NC \NR
\BC whatsit   \NC when hyphenationbounds 1 or 3 \NC \NR
\BC glue      \NC yes \NC \NR
\BC math      \NC skipped \NC \NR
\BC glyph     \NC exhyphenchar (one only) : yes (so no -- ---) \NC \NR
\BC otherwise \NC yes \NC \NR
\LL
\stoptabulate

The word end is determined as follows:

\starttabulate[|l|l|]
\DB node      \BC behaviour \NC \NR
\TB
\BC boundary  \NC yes \NC \NR
\BC glyph     \NC yes when different language \NC \NR
\BC glue      \NC yes \NC \NR
\BC penalty   \NC yes \NC \NR
\BC kern      \NC yes when not italic (for some historic reason) \NC \NR
\BC hlist     \NC when hyphenationbounds 2 or 3 \NC \NR
\BC vlist     \NC when hyphenationbounds 2 or 3 \NC \NR
\BC rule      \NC when hyphenationbounds 2 or 3 \NC \NR
\BC dir       \NC when hyphenationbounds 2 or 3 \NC \NR
\BC whatsit   \NC when hyphenationbounds 2 or 3 \NC \NR
\BC ins       \NC when hyphenationbounds 2 or 3 \NC \NR
\BC adjust    \NC when hyphenationbounds 2 or 3 \NC \NR
\LL
\stoptabulate

\in {Figures} [hb:1] upto \in [hb:5] show some examples. In all cases we set the
min values to 1 and make sure that the words hyphenate at each character.

\hyphenation{o-n-e t-w-o}

\def\SomeTest#1#2%
  {\lefthyphenmin  \plusone
   \righthyphenmin \plusone
   \parindent      \zeropoint
   \everypar       \emptytoks
   \dontcomplain
   \hbox to 2cm {%
     \vtop {%
       \hsize 1pt
       \hyphenationbounds#1
       #2
       \par}}}

\startplacefigure[reference=hb:1,title={\type{one}}]
    \startcombination[4*1]
        {\SomeTest{0}{one}} {\type{0}}
        {\SomeTest{1}{one}} {\type{1}}
        {\SomeTest{2}{one}} {\type{2}}
        {\SomeTest{3}{one}} {\type{3}}
    \stopcombination
\stopplacefigure

\startplacefigure[reference=hb:2,title={\type{one\null two}}]
    \startcombination[4*1]
        {\SomeTest{0}{one\null two}} {\type{0}}
        {\SomeTest{1}{one\null two}} {\type{1}}
        {\SomeTest{2}{one\null two}} {\type{2}}
        {\SomeTest{3}{one\null two}} {\type{3}}
    \stopcombination
\stopplacefigure

\startplacefigure[reference=hb:3,title={\type{\null one\null two}}]
    \startcombination[4*1]
        {\SomeTest{0}{\null one\null two}} {\type{0}}
        {\SomeTest{1}{\null one\null two}} {\type{1}}
        {\SomeTest{2}{\null one\null two}} {\type{2}}
        {\SomeTest{3}{\null one\null two}} {\type{3}}
    \stopcombination
\stopplacefigure

\startplacefigure[reference=hb:4,title={\type{one\null two\null}}]
    \startcombination[4*1]
        {\SomeTest{0}{one\null two\null}} {\type{0}}
        {\SomeTest{1}{one\null two\null}} {\type{1}}
        {\SomeTest{2}{one\null two\null}} {\type{2}}
        {\SomeTest{3}{one\null two\null}} {\type{3}}
    \stopcombination
\stopplacefigure

\startplacefigure[reference=hb:5,title={\type{\null one\null two\null}}]
    \startcombination[4*1]
        {\SomeTest{0}{\null one\null two\null}} {\type{0}}
        {\SomeTest{1}{\null one\null two\null}} {\type{1}}
        {\SomeTest{2}{\null one\null two\null}} {\type{2}}
        {\SomeTest{3}{\null one\null two\null}} {\type{3}}
    \stopcombination
\stopplacefigure

% (Future versions of \LUATEX\ might provide more granularity.)

In traditional \TEX\ ligature building and hyphenation are interwoven with the
line break mechanism. In \LUATEX\ these phases are isolated. As a consequence we
deal differently with (a sequence of) explicit hyphens. We already have added
some control over aspects of the hyphenation and yet another one concerns
automatic hyphens (e.g.\ \type {-} characters in the input).

When \lpr {automatichyphenmode} has a value of 0, a hyphen will be turned into
an automatic discretionary. The snippets before and after it will not be
hyphenated. A side effect is that a leading hyphen can lead to a split but one
will seldom run into that situation. Setting a pre and post character makes this
more prominent. A value of 1 will prevent this side effect and a value of 2 will
not turn the hyphen into a discretionary. Experiments with other options, like
permitting hyphenation of the words on both sides were discarded.

\startbuffer[a]
before-after \par
before--after \par
before---after \par
\stopbuffer

\startbuffer[b]
-before \par
after- \par
--before \par
after-- \par
---before \par
after--- \par
\stopbuffer

\startbuffer[c]
before-after \par
before--after \par
before---after \par
\stopbuffer

\startbuffer[demo]
\startcombination[nx=4,ny=3,location=top]
    {\framed[align=normal,strut=no,top=\vskip.5ex,bottom=\vskip.5ex]{\automatichyphenmode\zerocount \hsize6em \getbuffer[a]}} {A~0~6em}
    {\framed[align=normal,strut=no,top=\vskip.5ex,bottom=\vskip.5ex]{\automatichyphenmode\zerocount \hsize2pt \getbuffer[a]}} {A~0~2pt}
    {\framed[align=normal,strut=no,top=\vskip.5ex,bottom=\vskip.5ex]{\automatichyphenmode\plusone   \hsize2pt \getbuffer[a]}} {A~1~2pt}
    {\framed[align=normal,strut=no,top=\vskip.5ex,bottom=\vskip.5ex]{\automatichyphenmode\plustwo   \hsize2pt \getbuffer[a]}} {A~2~2pt}
    {\framed[align=normal,strut=no,top=\vskip.5ex,bottom=\vskip.5ex]{\automatichyphenmode\zerocount \hsize6em \getbuffer[b]}} {B~0~6em}
    {\framed[align=normal,strut=no,top=\vskip.5ex,bottom=\vskip.5ex]{\automatichyphenmode\zerocount \hsize2pt \getbuffer[b]}} {B~0~2pt}
    {\framed[align=normal,strut=no,top=\vskip.5ex,bottom=\vskip.5ex]{\automatichyphenmode\plusone   \hsize2pt \getbuffer[b]}} {B~1~2pt}
    {\framed[align=normal,strut=no,top=\vskip.5ex,bottom=\vskip.5ex]{\automatichyphenmode\plustwo   \hsize2pt \getbuffer[b]}} {B~2~2pt}
    {\framed[align=normal,strut=no,top=\vskip.5ex,bottom=\vskip.5ex]{\automatichyphenmode\zerocount \hsize6em \getbuffer[c]}} {C~0~6em}
    {\framed[align=normal,strut=no,top=\vskip.5ex,bottom=\vskip.5ex]{\automatichyphenmode\zerocount \hsize2pt \getbuffer[c]}} {C~0~2pt}
    {\framed[align=normal,strut=no,top=\vskip.5ex,bottom=\vskip.5ex]{\automatichyphenmode\plusone   \hsize2pt \getbuffer[c]}} {C~1~2pt}
    {\framed[align=normal,strut=no,top=\vskip.5ex,bottom=\vskip.5ex]{\automatichyphenmode\plustwo   \hsize2pt \getbuffer[c]}} {C~2~2pt}
\stopcombination
\stopbuffer

\startplacefigure[reference=automatichyphenmode:1,title={The automatic modes \type {0} (default), \type {1} and \type {2}, with a \prm {hsize}
of 6em and 2pt (which triggers a linebreak).}]
    \dontcomplain \tt \getbuffer[demo]
\stopplacefigure

\startplacefigure[reference=automatichyphenmode:2,title={The automatic modes \type {0} (default), \type {1} and \type {2}, with \lpr {preexhyphenchar} and \lpr {postexhyphenchar} set to characters \type {A} and \type {B}.}]
    \postexhyphenchar`A\relax
    \preexhyphenchar `B\relax
    \dontcomplain \tt \getbuffer[demo]
\stopplacefigure

In \in {figure} [automatichyphenmode:1] \in {and} [automatichyphenmode:2] we show
what happens with three samples:

Input A: \typebuffer[a]
Input B: \typebuffer[b]
Input C: \typebuffer[c]

As with primitive companions of other single character commands, the \prm {-}
command has a more verbose primitive version in \lpr {explicitdiscretionary}
and the normally intercepted in the hyphenator character \type {-} (or whatever
is configured) is available as \lpr {automaticdiscretionary}.

\stopsection

\startsection[title={The main control loop}]

\topicindex {main loop}
\topicindex {hyphenation}

In \LUATEX's main loop, almost all input characters that are to be typeset are
converted into \nod {glyph} node records with subtype \quote {character}, but
there are a few exceptions.

\startitemize[n]

\startitem
    The \prm {accent} primitive creates nodes with subtype \quote {glyph}
    instead of \quote {character}: one for the actual accent and one for the
    accentee. The primary reason for this is that \prm {accent} in \TEX82 is
    explicitly dependent on the current font encoding, so it would not make much
    sense to attach a new meaning to the primitive's name, as that would
    invalidate many old documents and macro packages. A secondary reason is that
    in \TEX82, \prm {accent} prohibits hyphenation of the current word. Since
    in \LUATEX\ hyphenation only takes place on \quote {character} nodes, it is
    possible to achieve the same effect. Of course, modern \UNICODE\ aware macro
    packages will not use the \prm {accent} primitive at all but try to map
    directly on composed characters.

    This change of meaning did happen with \prm {char}, that now generates
    \quote {glyph} nodes with a character subtype. In traditional \TEX\ there was
    a strong relationship between the 8|-|bit input encoding, hyphenation and
    glyphs taken from a font. In \LUATEX\ we have \UTF\ input, and in most cases
    this maps directly to a character in a font, apart from glyph replacement in
    the font engine. If you want to access arbitrary glyphs in a font directly
    you can always use \LUA\ to do so, because fonts are available as \LUA\
    table.
\stopitem

\startitem
    All the results of processing in math mode eventually become nodes with
    \quote {glyph} subtypes. In fact, the result of processing math is just
    a regular list of glyphs, kerns, glue, penalties, boxes etc.
\stopitem

\startitem
    The \ALEPH|-|derived commands \lpr {leftghost} and \lpr {rightghost}
    create nodes of a third subtype: \quote {ghost}. These nodes are ignored
    completely by all further processing until the stage where inter|-|glyph
    kerning is added.
\stopitem

\startitem
    Automatic discretionaries are handled differently. \TEX82 inserts an empty
    discretionary after sensing an input character that matches the \prm
    {hyphenchar} in the current font. This test is wrong in our opinion: whether
    or not hyphenation takes place should not depend on the current font, it is a
    language property. \footnote {When \TEX\ showed up we didn't have \UNICODE\
    yet and being limited to eight bits meant that one sometimes had to
    compromise between supporting character input, glyph rendering, hyphenation.}

    In \LUATEX, it works like this: if \LUATEX\ senses a string of input
    characters that matches the value of the new integer parameter \prm
    {exhyphenchar}, it will insert an explicit discretionary after that series of
    nodes. Initially \TEX\ sets the \type {\exhyphenchar=`\-}. Incidentally, this
    is a global parameter instead of a language-specific one because it may be
    useful to change the value depending on the document structure instead of the
    text language.

    The insertion of discretionaries after a sequence of explicit hyphens happens
    at the same time as the other hyphenation processing, {\it not\/} inside the
    main control loop.

    The only use \LUATEX\ has for \prm {hyphenchar} is at the check whether a
    word should be considered for hyphenation at all. If the \prm {hyphenchar}
    of the font attached to the first character node in a word is negative, then
    hyphenation of that word is abandoned immediately. This behaviour is added
    for backward compatibility only, and the use of \type {\hyphenchar=-1} as a
    means of preventing hyphenation should not be used in new \LUATEX\ documents.
\stopitem

\startitem
    The \prm {setlanguage} command no longer creates whatsits. The meaning of
    \prm {setlanguage} is changed so that it is now an integer parameter like all
    others. That integer parameter is used in \type {\glyph_node} creation to add
    language information to the glyph nodes. In conjunction, the \prm {language}
    primitive is extended so that it always also updates the value of \prm
    {setlanguage}.
\stopitem

\startitem
    The \prm {noboundary} command (that prohibits word boundary processing
    where that would normally take place) now does create nodes. These nodes are
    needed because the exact place of the \prm {noboundary} command in the
    input stream has to be retained until after the ligature and font processing
    stages.
\stopitem

\startitem
    There is no longer a \type {main_loop} label in the code. Remember that
    \TEX82 did quite a lot of processing while adding \type {char_nodes} to the
    horizontal list? For speed reasons, it handled that processing code outside
    of the \quote {main control} loop, and only the first character of any \quote
    {word} was handled by that \quote {main control} loop. In \LUATEX, there is
    no longer a need for that (all hard work is done later), and the (now very
    small) bits of character|-|handling code have been moved back inline. When
    \prm {tracingcommands} is on, this is visible because the full word is
    reported, instead of just the initial character.
\stopitem

\stopitemize

Because we tend to make hard coded behaviour configurable a few new primitives
have been added:

\starttyping
\hyphenpenaltymode
\automatichyphenpenalty
\explicithyphenpenalty
\stoptyping

The first parameter has the following consequences for automatic discs (the ones
resulting from an \prm {exhyphenchar}:

\starttabulate[|c|l|l|]
\DB mode     \BC automatic disc \type {-}      \BC explicit disc \prm{-}         \NC \NR
\TB
\NC \type{0} \NC \prm {exhyphenpenalty}        \NC \prm {exhyphenpenalty}        \NC \NR
\NC \type{1} \NC \prm {hyphenpenalty}          \NC \prm {hyphenpenalty}          \NC \NR
\NC \type{2} \NC \prm {exhyphenpenalty}        \NC \prm {hyphenpenalty}          \NC \NR
\NC \type{3} \NC \prm {hyphenpenalty}          \NC \prm {exhyphenpenalty}        \NC \NR
\NC \type{4} \NC \lpr {automatichyphenpenalty} \NC \lpr {explicithyphenpenalty}  \NC \NR
\NC \type{5} \NC \prm {exhyphenpenalty}        \NC \lpr {explicithyphenpenalty}  \NC \NR
\NC \type{6} \NC \prm {hyphenpenalty}          \NC \lpr {explicithyphenpenalty}  \NC \NR
\NC \type{7} \NC \lpr {automatichyphenpenalty} \NC \prm {exhyphenpenalty}        \NC \NR
\NC \type{8} \NC \lpr {automatichyphenpenalty} \NC \prm {hyphenpenalty}          \NC \NR
\LL
\stoptabulate

other values do what we always did in \LUATEX: insert \prm {exhyphenpenalty}.

\stopsection

\startsection[title={Loading patterns and exceptions},reference=patternsexceptions]

\topicindex {hyphenation}
\topicindex {hyphenation+patterns}
\topicindex {hyphenation+exceptions}
\topicindex {patterns}
\topicindex {exceptions}

Although we keep the traditional approach towards hyphenation (which is still
superior) the implementation of the hyphenation algorithm in \LUATEX\ is quite
different from the one in \TEX82.

After expansion, the argument for \prm {patterns} has to be proper \UTF8 with
individual patterns separated by spaces, no \prm {char} or \prm {chardef}d
commands are allowed. The current implementation is quite strict and will reject
all non|-|\UNICODE\ characters. Likewise, the expanded argument for \prm
{hyphenation} also has to be proper \UTF8, but here a bit of extra syntax is
provided:

\startitemize[n]
\startitem
    Three sets of arguments in curly braces (\type {{}{}{}}) indicate a desired
    complex discretionary, with arguments as in \prm {discretionary}'s command in
    normal document input.
\stopitem
\startitem
    A \type {-} indicates a desired simple discretionary, cf.\ \type {\-} and
    \type {\discretionary{-}{}{}} in normal document input.
\stopitem
\startitem
    Internal command names are ignored. This rule is provided especially for \prm
    {discretionary}, but it also helps to deal with \prm {relax} commands that
    may sneak in.
\stopitem
\startitem
    An \type {=} indicates a (non|-|discretionary) hyphen in the document input.
\stopitem
\stopitemize

The expanded argument is first converted back to a space|-|separated string while
dropping the internal command names. This string is then converted into a
dictionary by a routine that creates key|-|value pairs by converting the other
listed items. It is important to note that the keys in an exception dictionary
can always be generated from the values. Here are a few examples:

\starttabulate[|l|l|l|]
\DB value                  \BC implied key (input) \BC effect \NC\NR
\TB
\NC \type {ta-ble}         \NC table               \NC \type {ta\-ble} ($=$ \type {ta\discretionary{-}{}{}ble}) \NC\NR
\NC \type {ba{k-}{}{c}ken} \NC backen              \NC \type {ba\discretionary{k-}{}{c}ken} \NC\NR
\LL
\stoptabulate

The resultant patterns and exception dictionary will be stored under the language
code that is the present value of \prm {language}.

In the last line of the table, you see there is no \prm {discretionary} command
in the value: the command is optional in the \TEX-based input syntax. The
underlying reason for that is that it is conceivable that a whole dictionary of
words is stored as a plain text file and loaded into \LUATEX\ using one of the
functions in the \LUA\ \type {lang} library. This loading method is quite a bit
faster than going through the \TEX\ language primitives, but some (most?) of that
speed gain would be lost if it had to interpret command sequences while doing so.

It is possible to specify extra hyphenation points in compound words by using
\type {{-}{}{-}} for the explicit hyphen character (replace \type {-} by the
actual explicit hyphen character if needed). For example, this matches the word
\quote {multi|-|word|-|boundaries} and allows an extra break inbetween \quote
{boun} and \quote {daries}:

\starttyping
\hyphenation{multi{-}{}{-}word{-}{}{-}boun-daries}
\stoptyping

The motivation behind the \ETEX\ extension \prm {savinghyphcodes} was that
hyphenation heavily depended on font encodings. This is no longer true in
\LUATEX, and the corresponding primitive is basically ignored. Because we now
have \lpr {hjcode}, the case relate codes can be used exclusively for \prm
{uppercase} and \prm {lowercase}.

The three curly brace pair pattern in an exception can be somewhat unexpected so
we will try to explain it by example. The pattern \type {foo{}{}{x}bar} pattern
creates a lookup \type {fooxbar} and the pattern \type {foo{}{}{}bar} creates
\type {foobar}. Then, when a hit happens there is a replacement text (\type {x})
or none. Because we introduced penalties in discretionary nodes, the exception
syntax now also can take a penalty specification. The value between square brackets
is a multiplier for \lpr {exceptionpenalty}. Here we have set it to 10000 so
effectively we get 30000 in the example.

\def\ShowSample#1#2%
  {\startlinecorrection[blank]
   \hyphenation{#1}%
   \exceptionpenalty=10000
   \bTABLE[foregroundstyle=type]
     \bTR
       \bTD[align=middle,nx=4] \type{#1} \eTD
     \eTR
     \bTR
       \bTD[align=middle] \type{10em} \eTD
       \bTD[align=middle] \type {3em} \eTD
       \bTD[align=middle] \type {0em} \eTD
       \bTD[align=middle] \type {6em} \eTD
     \eTR
     \bTR
       \bTD[width=10em]\vtop{\hsize 10em 123 #2 123\par}\eTD
       \bTD[width=10em]\vtop{\hsize  3em 123 #2 123\par}\eTD
       \bTD[width=10em]\vtop{\hsize  0em 123 #2 123\par}\eTD
       \bTD[width=10em]\vtop{\setupalign[verytolerant,stretch]\rmtf\hsize 6em 123 #2 #2 #2 #2 123\par}\eTD
     \eTR
   \eTABLE
   \stoplinecorrection}

\ShowSample{x{a-}{-b}{}x{a-}{-b}{}x{a-}{-b}{}x{a-}{-b}{}xx}{xxxxxx}
\ShowSample{x{a-}{-b}{}x{a-}{-b}{}[3]x{a-}{-b}{}[1]x{a-}{-b}{}xx}{xxxxxx}

\ShowSample{z{a-}{-b}{z}{a-}{-b}{z}{a-}{-b}{z}{a-}{-b}{z}z}{zzzzzz}
\ShowSample{z{a-}{-b}{z}{a-}{-b}{z}[3]{a-}{-b}{z}[1]{a-}{-b}{z}z}{zzzzzz}

\stopsection

\startsection[title={Applying hyphenation}]

\topicindex {hyphenation+how it works}
\topicindex {hyphenation+discretionaries}
\topicindex {discretionaries}

The internal structures \LUATEX\ uses for the insertion of discretionaries in
words is very different from the ones in \TEX82, and that means there are some
noticeable differences in handling as well.

First and foremost, there is no \quote {compressed trie} involved in hyphenation.
The algorithm still reads pattern files generated by \PATGEN, but \LUATEX\ uses a
finite state hash to match the patterns against the word to be hyphenated. This
algorithm is based on the \quote {libhnj} library used by \OPENOFFICE, which in
turn is inspired by \TEX.

There are a few differences between \LUATEX\ and \TEX82 that are a direct result
of the implementation:

\startitemize
\startitem
    \LUATEX\ happily hyphenates the full \UNICODE\ character range.
\stopitem
\startitem
    Pattern and exception dictionary size is limited by the available memory
    only, all allocations are done dynamically. The trie|-|related settings in
    \type {texmf.cnf} are ignored.
\stopitem
\startitem
    Because there is no \quote {trie preparation} stage, language patterns never
    become frozen. This means that the primitive \prm {patterns} (and its \LUA\
    counterpart \type {lang.patterns}) can be used at any time, not only in
    ini\TEX.
\stopitem
\startitem
    Only the string representation of \prm {patterns} and \prm {hyphenation} is
    stored in the format file. At format load time, they are simply
    re|-|evaluated. It follows that there is no real reason to preload languages
    in the format file. In fact, it is usually not a good idea to do so. It is
    much smarter to load patterns no sooner than the first time they are actually
    needed.
\stopitem
\startitem
    \LUATEX\ uses the language-specific variables \lpr {prehyphenchar} and \lpr
    {posthyphenchar} in the creation of implicit discretionaries, instead of
    \TEX82's \prm {hyphenchar}, and the values of the language|-|specific
    variables \lpr {preexhyphenchar} and \lpr {postexhyphenchar} for explicit
    discretionaries (instead of \TEX82's empty discretionary).
\stopitem
\startitem
    The value of the two counters related to hyphenation, \prm {hyphenpenalty}
    and \prm {exhyphenpenalty}, are now stored in the discretionary nodes. This
    permits a local overload for explicit \prm {discretionary} commands. The
    value current when the hyphenation pass is applied is used. When no callbacks
    are used this is compatible with traditional \TEX. When you apply the \LUA\
    \type {lang.hyphenate} function the current values are used.
\stopitem
\startitem
    The hyphenation exception dictionary is maintained as key|-|value hash, and
    that is also dynamic, so the \type {hyph_size} setting is not used either.
\stopitem
\stopitemize

Because we store penalties in the disc node the \prm {discretionary} command has
been extended to accept an optional penalty specification, so you can do the
following:

\startbuffer
\hsize1mm
1:foo{\hyphenpenalty 10000\discretionary{}{}{}}bar\par
2:foo\discretionary penalty 10000 {}{}{}bar\par
3:foo\discretionary{}{}{}bar\par
\stopbuffer

\typebuffer

This results in:

\blank \start \getbuffer \stop \blank

Inserted characters and ligatures inherit their attributes from the nearest glyph
node item (usually the preceding one, but the following one for the items
inserted at the left-hand side of a word).

Word boundaries are no longer implied by font switches, but by language switches.
One word can have two separate fonts and still be hyphenated correctly (but it
can not have two different languages, the \prm {setlanguage} command forces a
word boundary).

All languages start out with \type {\prehyphenchar=`\-}, \type {\posthyphenchar=0},
\type {\preexhyphenchar=0} and \type {\postexhyphenchar=0}. When you assign the
values of one of these four parameters, you are actually changing the settings
for the current \prm {language}, this behaviour is compatible with \prm {patterns}
and \prm {hyphenation}.

\LUATEX\ also hyphenates the first word in a paragraph. Words can be up to 256
characters long (up from 64 in \TEX82). Longer words are ignored right now, but
eventually either the limitation will be removed or perhaps it will become
possible to silently ignore the excess characters (this is what happens in
\TEX82, but there the behaviour cannot be controlled).

If you are using the \LUA\ function \type {lang.hyphenate}, you should be aware
that this function expects to receive a list of \quote {character} nodes. It will
not operate properly in the presence of \quote {glyph}, \quote {ligature}, or
\quote {ghost} nodes, nor does it know how to deal with kerning.

\stopsection

\startsection[title={Applying ligatures and kerning}]

\topicindex {ligatures}
\topicindex {kerning}

After all possible hyphenation points have been inserted in the list, \LUATEX\
will process the list to convert the \quote {character} nodes into \quote {glyph}
and \quote {ligature} nodes. This is actually done in two stages: first all
ligatures are processed, then all kerning information is applied to the result
list. But those two stages are somewhat dependent on each other: If the used font
makes it possible to do so, the ligaturing stage adds virtual \quote {character}
nodes to the word boundaries in the list. While doing so, it removes and
interprets \prm {noboundary} nodes. The kerning stage deletes those word
boundary items after it is done with them, and it does the same for \quote
{ghost} nodes. Finally, at the end of the kerning stage, all remaining \quote
{character} nodes are converted to \quote {glyph} nodes.

This word separation is worth mentioning because, if you overrule from \LUA\ only
one of the two callbacks related to font handling, then you have to make sure you
perform the tasks normally done by \LUATEX\ itself in order to make sure that the
other, non|-|overruled, routine continues to function properly.

Although we could improve the situation the reality is that in modern \OPENTYPE\
fonts ligatures can be constructed in many ways: by replacing a sequence of
characters by one glyph, or by selectively replacing individual glyphs, or by
kerning, or any combination of this. Add to that contextual analysis and it will
be clear that we have to let \LUA\ do that job instead. The generic font handler
that we provide (which is part of \CONTEXT) distinguishes between base mode
(which essentially is what we describe here and which delegates the task to \TEX)
and node mode (which deals with more complex fonts.

Let's look at an example. Take the word \type {office}, hyphenated \type
{of-fice}, using a \quote {normal} font with all the \type {f}-\type {f} and
\type {f}-\type {i} type ligatures:

\starttabulate[|l|l|]
\NC initial              \NC \type {{o}{f}{f}{i}{c}{e}}             \NC\NR
\NC after hyphenation    \NC \type {{o}{f}{{-},{},{}}{f}{i}{c}{e}}  \NC\NR
\NC first ligature stage \NC \type {{o}{{f-},{f},{<ff>}}{i}{c}{e}}  \NC\NR
\NC final result         \NC \type {{o}{{f-},{<fi>},{<ffi>}}{c}{e}} \NC\NR
\stoptabulate

That's bad enough, but let us assume that there is also a hyphenation point
between the \type {f} and the \type {i}, to create \type {of-f-ice}. Then the
final result should be:

\starttyping
{o}{{f-},
    {{f-},
     {i},
     {<fi>}},
    {{<ff>-},
     {i},
     {<ffi>}}}{c}{e}
\stoptyping

with discretionaries in the post-break text as well as in the replacement text of
the top-level discretionary that resulted from the first hyphenation point.

Here is that nested solution again, in a different representation:

\testpage[4]

\starttabulate[|l|c|c|c|c|c|c|]
\DB         \BC pre           \BC     \BC post      \BC       \BC replace       \BC       \NC \NR
\TB
\NC topdisc \NC \type {f-}    \NC (1) \NC           \NC sub 1 \NC               \NC sub 2 \NC \NR
\NC sub 1   \NC \type {f-}    \NC (2) \NC \type {i} \NC (3)   \NC \type {<fi>}  \NC (4)   \NC \NR
\NC sub 2   \NC \type {<ff>-} \NC (5) \NC \type {i} \NC (6)   \NC \type {<ffi>} \NC (7)   \NC \NR
\LL
\stoptabulate

When line breaking is choosing its breakpoints, the following fields will
eventually be selected:

\starttabulate[|l|c|c|]
\NC \type {of-f-ice} \NC \type {f-}    \NC (1) \NC \NR
\NC                  \NC \type {f-}    \NC (2) \NC \NR
\NC                  \NC \type {i}     \NC (3) \NC \NR
\NC \type {of-fice}  \NC \type {f-}    \NC (1) \NC \NR
\NC                  \NC \type {<fi>}  \NC (4) \NC \NR
\NC \type {off-ice}  \NC \type {<ff>-} \NC (5) \NC \NR
\NC                  \NC \type {i}     \NC (6) \NC \NR
\NC \type {office}   \NC \type {<ffi>} \NC (7) \NC \NR
\stoptabulate

The current solution in \LUATEX\ is not able to handle nested discretionaries,
but it is in fact smart enough to handle this fictional \type {of-f-ice} example.
It does so by combining two sequential discretionary nodes as if they were a
single object (where the second discretionary node is treated as an extension of
the first node).

One can observe that the \type {of-f-ice} and \type {off-ice} cases both end with
the same actual post replacement list (\type {i}), and that this would be the
case even if \type {i} was the first item of a potential following ligature like
\type {ic}. This allows \LUATEX\ to do away with one of the fields, and thus make
the whole stuff fit into just two discretionary nodes.

The mapping of the seven list fields to the six fields in this discretionary node
pair is as follows:

\starttabulate[|l|c|c|]
\DB field                 \BC description   \NC       \NC \NR
\TB
\NC \type {disc1.pre}     \NC \type {f-}    \NC (1)   \NC \NR
\NC \type {disc1.post}    \NC \type {<fi>}  \NC (4)   \NC \NR
\NC \type {disc1.replace} \NC \type {<ffi>} \NC (7)   \NC \NR
\NC \type {disc2.pre}     \NC \type {f-}    \NC (2)   \NC \NR
\NC \type {disc2.post}    \NC \type {i}     \NC (3,6) \NC \NR
\NC \type {disc2.replace} \NC \type {<ff>-} \NC (5)   \NC \NR
\LL
\stoptabulate

What is actually generated after ligaturing has been applied is therefore:

\starttyping
{o}{{f-},
    {<fi>},
    {<ffi>}}
   {{f-},
    {i},
    {<ff>-}}{c}{e}
\stoptyping

The two discretionaries have different subtypes from a discretionary appearing on
its own: the first has subtype 4, and the second has subtype 5. The need for
these special subtypes stems from the fact that not all of the fields appear in
their \quote {normal} location. The second discretionary especially looks odd,
with things like the \type {<ff>-} appearing in \type {disc2.replace}. The fact
that some of the fields have different meanings (and different processing code
internally) is what makes it necessary to have different subtypes: this enables
\LUATEX\ to distinguish this sequence of two joined discretionary nodes from the
case of two standalone discretionaries appearing in a row.

Of course there is still that relationship with fonts: ligatures can be implemented by
mapping a sequence of glyphs onto one glyph, but also by selective replacement and
kerning. This means that the above examples are just representing the traditional
approach.

\stopsection

\startsection[title={Breaking paragraphs into lines}]

\topicindex {linebreaks}
\topicindex {paragraphs}
\topicindex {discretionaries}

This code is almost unchanged, but because of the above|-|mentioned changes
with respect to discretionaries and ligatures, line breaking will potentially be
different from traditional \TEX. The actual line breaking code is still based on
the \TEX82 algorithms, and it does not expect there to be discretionaries inside
of discretionaries. But, as patterns evolve and font handling can influence
discretionaries, you need to be aware of the fact that long term consistency is not
an engine matter only.

But that situation is now fairly common in \LUATEX, due to the changes to the
ligaturing mechanism. And also, the \LUATEX\ discretionary nodes are implemented
slightly different from the \TEX82 nodes: the \type {no_break} text is now
embedded inside the disc node, where previously these nodes kept their place in
the horizontal list. In traditional \TEX\ the discretionary node contains a
counter indicating how many nodes to skip, but in \LUATEX\ we store the pre, post
and replace text in the discretionary node.

The combined effect of these two differences is that \LUATEX\ does not always use
all of the potential breakpoints in a paragraph, especially when fonts with many
ligatures are used. Of course kerning also complicates matters here.

\stopsection

\startsection[title={The \type {lang} library}][library=lang]

\subsection {\type {new} and \type {id}}

\topicindex {languages+library}

\libindex {new}
\libindex {id}

This library provides the interface to \LUATEX's structure representing a
language, and the associated functions.

\startfunctioncall
<language> l = lang.new()
<language> l = lang.new(<number> id)
\stopfunctioncall

This function creates a new userdata object. An object of type \type {<language>}
is the first argument to most of the other functions in the \type {lang} library.
These functions can also be used as if they were object methods, using the colon
syntax. Without an argument, the next available internal id number will be
assigned to this object. With argument, an object will be created that links to
the internal language with that id number.

\startfunctioncall
<number> n = lang.id(<language> l)
\stopfunctioncall

The number returned is the internal \prm {language} id number this object refers to.

\subsection {\type {hyphenation}}

\libindex {hyphenation}

You can hyphenate a string directly with:

\startfunctioncall
<string> n = lang.hyphenation(<language> l)
lang.hyphenation(<language> l, <string> n)
\stopfunctioncall

\subsection {\type {clear_hyphenation} and \type {clean}}

\libindex {clear_hyphenation}
\libindex {clean}

This either returns the current hyphenation exceptions for this language, or adds
new ones. The syntax of the string is explained in~\in {section}
[patternsexceptions].

\startfunctioncall
lang.clear_hyphenation(<language> l)
\stopfunctioncall

This call clears the exception dictionary (string) for this language.

\startfunctioncall
<string> n = lang.clean(<language> l, <string> o)
<string> n = lang.clean(<string> o)
\stopfunctioncall

This function creates a hyphenation key from the supplied hyphenation value. The
syntax of the argument string is explained in \in {section} [patternsexceptions].
This function is useful if you want to do something else based on the words in a
dictionary file, like spell|-|checking.

\subsection {\type {patterns} and \type {clear_patterns}}

\libindex {patterns}
\libindex {clear_patterns}

\startfunctioncall
<string> n = lang.patterns(<language> l)
lang.patterns(<language> l, <string> n)
\stopfunctioncall

This adds additional patterns for this language object, or returns the current
set. The syntax of this string is explained in \in {section}
[patternsexceptions].

\startfunctioncall
lang.clear_patterns(<language> l)
\stopfunctioncall

This can be used to clear the pattern dictionary for a language.

\subsection {\type {hyphenationmin}}

\libindex {hyphenationmin}

This function sets (or gets) the value of the \TEX\ parameter
\type {\hyphenationmin}.

\startfunctioncall
n = lang.hyphenationmin(<language> l)
lang.hyphenationmin(<language> l, <number> n)
\stopfunctioncall

\subsection {\type {[pre|post][ex|]hyphenchar}}

\libindex {prehyphenchar}
\libindex {posthyphenchar}
\libindex {preexhyphenchar}
\libindex {postexhyphenchar}

\startfunctioncall
<number> n = lang.prehyphenchar(<language> l)
lang.prehyphenchar(<language> l, <number> n)

<number> n = lang.posthyphenchar(<language> l)
lang.posthyphenchar(<language> l, <number> n)
\stopfunctioncall

These two are used to get or set the \quote {pre|-|break} and \quote
{post|-|break} hyphen characters for implicit hyphenation in this language. The
intial values are decimal 45 (hyphen) and decimal~0 (indicating emptiness).

\startfunctioncall
<number> n = lang.preexhyphenchar(<language> l)
lang.preexhyphenchar(<language> l, <number> n)

<number> n = lang.postexhyphenchar(<language> l)
lang.postexhyphenchar(<language> l, <number> n)
\stopfunctioncall

These gets or set the \quote {pre|-|break} and \quote {post|-|break} hyphen
characters for explicit hyphenation in this language. Both are initially
decimal~0 (indicating emptiness).

\subsection {\type {hyphenate}}

\libindex {hyphenate}

The next call inserts hyphenation points (discretionary nodes) in a node list. If
\type {tail} is given as argument, processing stops on that node. Currently,
\type {success} is always true if \type {head} (and \type {tail}, if specified)
are proper nodes, regardless of possible other errors.

\startfunctioncall
<boolean> success = lang.hyphenate(<node> head)
<boolean> success = lang.hyphenate(<node> head, <node> tail)
\stopfunctioncall

Hyphenation works only on \quote {characters}, a special subtype of all the glyph
nodes with the node subtype having the value \type {1}. Glyph modes with
different subtypes are not processed. See \in {section} [charsandglyphs] for
more details.

\subsection {\type {[set|get]hjcode}}

\libindex {sethjcode}
\libindex {gethjcode}

The following two commands can be used to set or query hj codes:

\startfunctioncall
lang.sethjcode(<language> l, <number> char, <number> usedchar)
<number> usedchar = lang.gethjcode(<language> l, <number> char)
\stopfunctioncall

When you set a hjcode the current sets get initialized unless the set was already
initialized due to \prm {savinghyphcodes} being larger than zero.

\stopsection

\stopchapter

\stopcomponent

% \parindent0pt \hsize=1.1cm
% 12-34-56 \par
% 12-34-\hbox{56} \par
% 12-34-\vrule width 1em height 1.5ex \par
% 12-\hbox{34}-56 \par
% 12-\vrule width 1em height 1.5ex-56 \par
% \hjcode`\1=`\1 \hjcode`\2=`\2 \hjcode`\3=`\3 \hjcode`\4=`\4 \vskip.5cm
% 12-34-56 \par
% 12-34-\hbox{56} \par
% 12-34-\vrule width 1em height 1.5ex \par
% 12-\hbox{34}-56 \par
% 12-\vrule width 1em height 1.5ex-56 \par

